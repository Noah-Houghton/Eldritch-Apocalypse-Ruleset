\documentclass[./././main.tex]{subfiles}
 
\begin{document}

\section{Crossing the Veil}
The Veil is thin, particularly in these times. Nowhere is it more perilously weakened than in the tenebrous depths of the earth. More oft than not along leylines, the currents of magical power which bind land to the beyond, Rifts and Tears in the Veil have opened up, from which Eldritch horrors pour through and draw their strength.

Intrepid adventurers equipped with the correct Eldritch equipment could find ways to seal (or tear further) these wounds in the membrane between worlds. But be wary: powers beyond this world are the jealous guardians of these places, and will do whatever is necessary to protect their interests.

\subsection{Tears}
Even the smallest rip between worlds allows creatures of darkest night to seep out of their immortal prisons. Tears are small holes in the Veil, unnoticed or passed over by more powerful Eldritch creatures, too small for them but not too small for a plethora of Eldritch creatures. It is through Tears that Lovecraftian horrors came to stalk the shadows long before the comets fell. 

Tears do not require special Eldritch equipment to seal. They usually attach themselves to a physical object within the limits of the space. This object becomes the portal between worlds, and only things which can fit into or through it may use it to pass (e.g. a doorway cannot be used to transport a large horse, a potting plant cannot be used to transport a humanoid person, etc.). Rifts can be closed by destroying the object which has been possessed. Doing so unleashes a burst of Eldritch energy which draws creatures of the night towards it, as well as raising the Monstrosity of any nearby creature.

Closing a Tear will not only prevent further Eldritch creatures from coming through it, but is also one of the few ways to gather Eldritch Essence, an essential ingredient for high-level Eldritch magic and alchemy. 

% experimental
One such use for this ingredient is in the "What-If" spell, which allows the party to experience a world in which a single decision was changed. The exact workings of this spell are left to the GM and players to decide, but can range from a question and answer session with the GM to a full-blown one-shot in that alternate timeline. The purpose of this mechanic is to provide a diegetic way for players and the GM to experience different worlds left un-discovered by the flow of gameplay.

\subsubsection{Rifts}
Rifts are much larger than Tears, typically about the size of a modest suburban home. Due to their larger size and accordingly larger guardians, Rifts tend to be more important -- and more dangerous -- sites of Eldritch activity.

There is a peculiar quality to Rifts which is not shared by their smaller brethren: at size, this particularly concentration of Eldritch matter allows one suitably trained in the art of Eldritch manipulation to use the energy burst from its closing to alter distinct moments in time.
% experimental
On the destruction of a rift, the GM rolls a single die to determine which player is selected. The selected player may then choose a single small-to-medium impact (determined at the discretion of the GM), binary decision in their past to alter. The other members of the party will remember the events of the current timeline, but no one else will. Once changed in this way, that point in the timeline becomes fixed, and cannot be changed again.
\subsection{Vortexes}
Massive holes between this world and the Void, Vortexes are the only places large enough to accommodate the bodies -- and armies -- of the most powerful Eldritch creatures. To close such portals would be an incredible feat, which would bring the world a step closer to purging the dark horrors of the night forever. 

Closing a Vortex allows a player (chosen in the same manner as they would be for a Rift) to change any event in their timeline. Because, practically speaking, this will not occur until the end of a campaign (and even then, the attempt could end in a TPK), it is recommended that this Wish be allowed more leeway than might otherwise be given.
\end{document}