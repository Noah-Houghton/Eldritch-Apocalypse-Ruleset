% change '././main.tex' below to reflect sub-document's position in folder hierarchy
\documentclass[././main.tex]{subfiles}
 
\begin{document}

\begin{dndclass}{Eldritch Host}{Eldritch Hosts are those whose connection with nature extends to even those most unnatural creatures who roam the darkened streets. In exchange for the Eldritch creature's assistance in battle and otherwise, the Host provides its flesh and energy as sustenance. This is not always an easy partnership, nor always a willing one, but it is nevertheless powerful. Eldritch creatures' motivations are as diverse as they are powerful. Perhaps a Lawful Good Host contracts with a creature determined to defeat the Eldritch legions, or a Chaotic Evil host shares a love of destruction with her familiar.}
\narrative{"So it's done then?" Archon growled, rolling his shoulders as he tried to get used to the unsettling feeling of something crawling just beneath the skin. The creature's response reverberated in his head, less an actual response than a wave of emotion, a sense of dark satisfaction.
\newline
"I'll take that as a yes."
\newline
Archon flexed again and watched as the creature's terribly jagged, insect-like forearm materialized from the shadows above his shoulder, hanging down like some abominable shoulder piece. A grin stretched across his face, unnatural.
\newline
"Well then. Let's tear some shit up."
\newline
A satisfied, unearthly growl precipitated Archon's purposeful stride back towards the House of Sorrows.}

\classfeature[Every time you score a critical hit while in combat, you immediately regain a point of Eldritch Energy.|If you kill someone with an Eldritch ability, roll 1d8. If it is greater than 6, regain a point of Eldritch Energy as well as 5 additional points of Monstrosity.]{Eldritch Affinity}
{Starting when you choose this archetype at 3rd level, you gain special affinity with an Eldritch entity of your choice (at the GM's discretion). You immediately gain a number of Eldritch Energy points equal to your Wisdom modifier (minimum of 1). You can spend Eldritch Energy to perform a variety of Eldritch abilities.}

\classaction{Eldritch Assistance}{As a bonus action, you may spend a point of Eldritch Energy to summon the assistance of the Eldritch creature you are currently hosting. Its assistance allows you to add an additional 1d6 damage to a successful attack. This number increases by 1d10 every two levels in Eldritch Host you gain above 3rd level. The damage type is determined by the Eldritch creature you have contracted with. Using this ability also increases your Monstrosity by 5 points.}

\classaction{Eldritch Guardian}{Starting at 5th level, you may spend a point of Eldritch Energy as a reaction to reduce incoming damage by half. At 9th level, your Eldritch companion may make an additional reaction. This reaction cannot be spent on this ability.}


\classaction{Manifest}{Starting at 7th level, you may spend two points of Eldritch Energy to summon a spectral version of your Eldritch familiar. This specter has 5d6 HP and has a reduced set of abilities from its corporeal version. The specter lasts until the caster's concentration is broken, the spell is ended by the caster, or after an hour, whichever comes first.

At 17th level, this spell can be cast as a bonus action and no longer requires concentration to maintain. In addition, the specter gains an additional 4d6 HP and has full access to its abilities.}

\classaction{Eldritch Form}{A most terrible sight - twisted and snarling, abomination in two worlds. Starting at 11th level, you may spend two Eldritch Energy points to use an action to meld your form with your Eldritch familiar's. You immediately heal 4d6 hit points and gain advantage on your next attack. A series of other effects occur based on the selected familiar, detailed in the Monsters section of this expansion. Your Monstrosity increases by 30 points immediately when you use this ability and is reduced by 10 when the form ends.

At 15th level, this transformation becomes faster and may be performed as a bonus action.}

\end{dndclass}

\end{document}