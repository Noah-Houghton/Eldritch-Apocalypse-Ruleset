% change '././main.tex' below to reflect sub-document's position in folder hierarchy
\documentclass[././main.tex]{subfiles}
 
\begin{document}

\begin{dndclass}{Order of the Abomination}{Blood Hunters, already a rare and oft-untrusted group, are town heroes compared to the treatment of those among the Order of the Abomination. These creatures once flesh and blood took into their form the essence of the dark forms which plague this world, becoming something more powerful at a terrible cost. In the course of this ritual, the blood hunter merges their being with that of a specific eldritch creature and takes on a variety of their physiological and psychological peculiarities.}

\classaction{Blood of the Abomination}{By joining their blood with that of an abomination, the blood hunter permanently takes on some of the physical characteristics of their chosen creature. Because of this, they gain advantage on perception checks in their creature's natural environment. Further, they gain a resistance in accordance with their creature's unique physiology.}

\classaction{Monstrous Infusion}{In moments of need, the blood hunter can reach into their pool of eldritch energy and rejuvenate their tired and broken body in exchange for giving into their monstrous nature. Beginning at 3rd level, the blood hunter can recover 2d8 hit points while gaining 1 point of monstrosity, as a bonus action. For each two levels in blood hunter above level three, an additional 1d8 hit points is recovered. After two uses, the user must take a long rest before using the power again.}

\classaction{Sanguine Discipline}{Members of the Order of the Abomination are masters over their inner demons, and they have learned to use pain to sharpen their mind and harden their will as they seek to wrest control from their eldritch form. Beginning at 7th level, a member of the order can use an action to halve their monstrosity, rounding down, through meditative blood magic which inflicts 2d8 psychic damage. For each two levels above level 5, the damage increases by 1d8. This ability can be used once per long rest.}

\classaction{The Blood Quickens}{
At level 11, blood hunters from the order of the abomination can use a bonus action to dip into their eldritch energies to gain the speed of an eldritch abomination once per long rest. For up to a minute, they can use their bonus action to make an extra attack. However, at the beginning of each turn if the blood hunter has taken damage since the previous turn, they must make a wisdom saving throw with DC 15 or half the damage received that turn plus their current monstrosity, whichever is higher. On a failed saving throw, they gain a point of monstrosity.
}
\classaction{Nature of the Beast}{
Beginning at 15th level, members of this order learn to harness their monstrosity to produce one of following three effects. \\
\begin{itemize}
    \item Accursed Hide: Increase the AC of the hunter by their monstrosity divided by 4 rounded down.
    \item Unearthly Might: Increase the hunter's strength by half their monstrosity rounded down.
    \item Supernatural Swiftness: Increase the hunter's dexterity by half their monstrosity rounded down.
\end{itemize}
This effect can be used once per short rest and lasts until the next short rest.

}

\classaction{Succumb to the Beast}{
Starting at level 18, those trained in the order of the abomination can refuse death in favor of giving themselves fully over to their abomination. When reduced to 0 hit points or subject to an effect where they would be killed outright, they can choose to instead be revived to full health, taking on the full form of their eldritch taint with their monstrosity maxed out at 20 points. In this state, however, they have no concept of friend or foe and they direct their attacks randomly or at the DM's discretion. Each turn, they can make a DC 20 Wisdom saving throw to attempt to regain sane control over their mind. Upon a successful saving throw or unconsciousness, the hunter regains full control. This ability can be used once per long rest.
}
\end{dndclass}

\end{document}