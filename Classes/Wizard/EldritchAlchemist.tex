% change '././main.tex' below to reflect sub-document's position in folder hierarchy
\documentclass[././main.tex]{subfiles}
 
\begin{document}

\begin{dndclass}{Eldritch Alchemist}{The Eldritch Alchemist is a Wizard who, through careful study or reckless contract, has developed an acute understanding of the otherworldly creatures which haunt the streets. Through careful brewing of potions and Elixirs, as well as the preparation of powerful spells, these practitioners of magic are often obsessed with knowledge acquisition and will do almost anything to acquire it. Only the most steadfast of Eldritch Alchemists is able to resist the tantalizing whispers of darkest knowledge; those who do not often fall prey to the unholy machinations of creatures who live far beyond the flickering pale light of this reality.}
\narrative{"No!" A bottle of some putrid-looking liquid crashed against the corrugated metal of the shed's walls, the shards falling into a scattered heap onto the dusty wooden floor. The brownish-green fluid oozed down the wall slowly, as if reluctant to meet the floor.}

"It's wrong! It's all wrong!" Atsor the Foresworn was not a patient man. A brilliant man, perhaps the smartest elf in the city, but not patient. Cursed with genius enough to set goals but insufficient to accomplish them, his assistants knew to avoid the lab when his shouts of inchoate rage and the sounds of bottles breaking disrupted the cool stillness of an unscrupulous evening.
\newline
"There must be something...something I'm missing..." Half-mad with frustration and the rancor of science gone awry, Atsor paged furiously through the alchemic texts of his forebears, searching for his mistake. The crazed mumblings of a corrupted genius in this strange and dangerous new world often attract the attention of otherworldly and dangerous creatures, willing to exchange unsavory knowledge for sanguine sacrifice. So it was this night, with this elven miscreant.
\newline
A drop of liquid darkness welled up from the spine of the text, slithering its unholy way towards aged, trembling fingers. "What are you, then?" the alchemist squawked, his curiosity overriding the sense of dread welling up in his most primal being. "Come to laugh at an old fool?"
\newline
\narrative{The darkness seemed to shake its head, oozing away through the pages. The scientist flipped carefully through the pages, coming to rest where the darkness had coiled itself, like some terrible feline caricature, around a series of arcane symbols. Atsor grumbled, turning away with a huff. "Damnable creature, I've tried this already! The mixture remains unstable, it requires some sort of..." A sizzling sound broke through his protestations. He turned back to the book to see the darkness slithering away, off the table and into the shadows. He approached the table cautiously to see an addition to the symbols burned onto the page, smoke still rising from its edges.
\newline
"Yes...yes, of course! How could I not have seen it before? The mixture requires something...otherworldly." The alchemist turned to the shadows, where he could swear an eye peered out at him. Ecstatic with the sensation of the nearness of discovery, the foolish genius extended his hands in supplication. The darkness was all too happy to oblige.
\newline
The next day, Qu'cy fell to an onslaught from within. Survivors tell of a crazed laugh, a lab coat streaked with an oily blackness, and an elf whose very being sang with unholy power.
}


\classfeature{Dark Bargain}{You immediately gain a number of Eldritch Knowledge points equal to your Intelligence modifier. You gain an additional point of Knowledge for every two levels you gain above 3 in Eldritch Alchemist. You may spend a short rest to recover 1d4 / 2 Knowledge at the expense of 2d10 Monstrosity, or regain Knowledge equal to half your maximum Secrets with a long rest.}


\classfeature{Eldritch Alchemy}{You immediately gain the ability to brew Common Potions using standard materials as well as first-level Elixirs. Potions you brew and Elixirs you make have a 1d4 chance to be more powerful than the standard version of the potion (at the discretion of the DM). At 7th level you may brew second-level Elixirs and Uncommon Potions. At 13th level you may brew third-level Elixirs. At 15th level you may brew Rare Potions. At 19th level you may brew Legendary Potions.

You may also modify a potion or Elixir to be a Splash Potion. These are throwable items whose primary attribute is Wisdom. The throwables have a 30/60 ft range with a 10 foot circular effect radius.

You may amplify your concoctions using Eldritch Essence. To do so, gain a point of Monstrosity to summon Eldritch Essence or harvest it from the world. Use the below table to determine the effect it has on the concoction.}

\header{Potion Modifiers}
\begin{dndtable}
   	\textbf{d6}  & \textbf{Effect} \\
   	1  & The blast radius of this potion is doubled \\
   	2  & Add 2d4 to the effect of this potion \\
   	3  & The effect of this potion is lowered by 1d4 \\
   	4  & The duration of this potion is doubled \\
   	5  & The user of this potion takes 1d6 damage upon use \\
   	6  & The user of the potion gains advantage on saving throws for the duration of the potion \\
\end{dndtable}

\classaction{Knowledge is Power}{You may spend a point of Knowledge to make a psychic attack against a creature of your choice. This attack does 2d6 damage per point of Knowledge spent and cannot be blocked. No more than two points of Knowledge may be used for this ability at a time. If the target is killed by this attack, immediately regain a point of Knowledge.}


\classaction{Whispers of Madness}{At 9th level, you begin to understand the maddening nature of universal truth. You may spend your Knowledge to attempt to drive a creature Mad. This attack does 1d6 damage per point of Knowledge spent. The target must make a Wisdom saving throw (DC 12 + (Knowledge * 2)) or be Enraged. The creature remains Enraged for ten minutes or until the caster loses concentration.}

\classaction{Darkness Beckons}{At 12th level, you realize that what you know can be used to turn others to your cause. You may spend your Knowledge to attempt to control a creature. This attack does 3d6 damage per point of Knowledge spent. The target must make a Wisdom saving throw (DC 15 + (Knowledge)) or fall under your control. The creature remains under your control for (Knowledge * 5) minutes or until the caster releases them.}

\end{dndclass}

\end{document}